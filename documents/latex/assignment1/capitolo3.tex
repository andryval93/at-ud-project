\rhead{\textit{Problem design}}
\lhead{}
In this paragraph we will propose two proto-personas (i.e. two possible actors present in our application) and for each of them, we will propose a different scenario. Finally, for each scenario we will draw up the Claims table.

\subsection{Personas}
\subsubsection{Mario - Studente}
Mario is a student at the University of Salerno. He is 24 years old and studies computer science and he's been deaf since birth. At least 3 days a week, he goes to university to study and to meet friends for the correction of the exercises. He would like to both improve communication with them and try to teach the main notions of the ASL language.

\subsubsection{Luca - Studente}
Luca is an economics student who, in addition to studying, also spends his time helping a deaf friend. He was interested in the ASL language and therefore wants to start studying it from home. 

\subsection{Scenarios}
\subsubsection{Mario}
Mario's going to university. After meeting a friend and having a coffee, they decide to start studying. Although we have successfully completed all the exercises, he has doubts about the correctness of a process. He decides to ask his friend for help.  But there are communication difficulties, so the friend decides to do the exercise without justifying his actions. In this way the problem is only remedied.

so Mario remembers to have the ProSigne application and you'll use the text-to-speach function. With this feature, the two of them overcome the communication difficulties and dispel the various doubts about the exercise.

\subsubsection{Luca}
As a social student, he decided to improve his knowledge of the ASL language, after having encountered several problems of communication with a friend. He decides to start this process of improvement by doing several searches in the network, but finds problems in the use of the platforms, of which many are paid. 

decides to start this improvement process using the ProSign application. He starts with a review of the alphabet and known words and then moves on to new words or phrases. In the end, again through ProSign, he decides to video call a friend to test his skills.

\clearpage
\subsection{Claim}
\subsubsection{Mario}
\begin{table}[h]
	\begin{tabular}{|r|r|}
		\hline
		PRO                                              & CONTRO                               \\ \hline
		\multicolumn{1}{|l|}{Possibility of video calls} & \multicolumn{1}{l|}{No face-to-face} \\ \hline
		Facilitates communication                        & Possibility to forget the data       \\ \hline
		Auto-learned                                     & Internet connection for use          \\ \hline
		\multicolumn{1}{|l|}{Report of the improvements} & \multicolumn{1}{l|}{}                \\ \hline
	\end{tabular}
\end{table}

\subsubsection{Luca}
\begin{table}[h]
	\begin{tabular}{|r|r|}
		\hline
		PRO                                              & CONTRO                               \\ \hline
		\multicolumn{1}{|l|}{Possibility of video calls} & \multicolumn{1}{l|}{No face-to-face} \\ \hline
		Facilitates communication                        & Possibility to forget the data       \\ \hline
		Auto-learned                                     & Internet connection for use          \\ \hline
		\multicolumn{1}{|l|}{Report of the improvements} & \multicolumn{1}{l|}{}                \\ \hline
	\end{tabular}
\end{table}