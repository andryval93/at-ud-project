\rhead{\textit{1 PROJECT SCOPE}}
\lhead{}
\subsection{Presentation of the problem}
The area in which we operate concerns the sphere of health, in particular we will deal with problems related to hearing loss.\\
Hearing loss has several consequences, for example, in children, it can affect the ability to learn the spoken language. In adults it can cause difficulties related to the world of work, while in the elderly the consequences may be related to loneliness. Our goal is to go and analyze how hearing problems can affect people's lives, analyze different reactions and then offer a tool that goes against their needs.

\subsection{Planning the collection of information during contextual investigations}


We have provided for two types of questionnaire (see the following two paragraphs), aimed at our two types of target group: one to deaf people and one to friends and family. Two types of interviews were conducted, one in the field and one through online surveys. We interviewed 5 deaf people in the field and then wrote down all the relevant aspects useful for the design of the application. As for friends and family, we have prepared an online survey using the Google tool, "Google Docs".

\subsubsection{Field interview}
Before starting the interview we will introduce to the subjects the idea of our project and of the problems we are trying to solve, we will ask for the authorization to the treatment of the answers given to us and if they want their answers to remain anonymous or their consent to give us their data (at least name, surname and specialization). The interview is divided into three parts: the first part is very general about how they relate to others every day, the second part contains more specific questions that represent the problems we want to solve through our application, and finally a question about how they would relate to a software solution to the problems encountered. These are the questions we want to ask: 
\begin{itemize}
	\item How long have you been deaf?
	\item How do you communicate and/or access information at home?
	\item How do you communicate and/or access information with friends?
	\item In what environments is it difficult for you to understand other people?
	\item How do you communicate in a small-group setting like a discussion group?
	
	\item What other accommodations do you use at home (bed-shaker alarm, signal lights and alert
	systems, etc.)?
	\item Have you used interpreters in the past? If so, what has been your experience?
	\item Do you prefer that interpreters use signing only, signing in English word order, or no signing
	but mouthing and gestures to lip read?
	\item Do you usually speak for yourself or do you prefer the interpreter to voice for you?
	
	\item Have you used speech-to-text or text-to-speach services?
	\item If not, would you feel comfortable watching a laptop screen to read the lecture rather than
	listening and lip-reading the instructor?
\end{itemize}

At the end of each interview we left in our contacts in case they had other suggestions or wanted to be updated on the development of the application.

\subsubsection{Online Survey Interview}
Using many social channels, so as to have a high number of responses, we will ask users to respond to our online questionnaire. The questionaire is also divided into four parts: the first part contains general questions about the user to understand what kind of user has answered, the second part contains questions about the relationship with the deaf person, the third part about specific problems that may arise and that we want to solve with our solution, and finally a question about the willingness of the user to use a software solution to these problems. The questionnaire will have the following questions:
\begin{itemize}
	\item Are you male or female?
	\item How old are you?
	\item How often do you interact with a deaf person?
	\item Usually, such as communicating with deaf people?
	\item Is it hard to understand?
	\item is it hard to make yourself understood?
	\item Do you think that using an application can improve communication?
	\item Do you think that usingn a application can halp you to know ASL language?
	
\end{itemize}
