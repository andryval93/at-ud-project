\rhead{\textit{Specification of usability requirements identified during analysis and design}}
\lhead{}
The usability principles find their natural collocation if they are 'cascaded' into the development life cycle of a system.

\subsection{Identifying the user's specifications}
it is a fundamental phase whose conduct requires the use of tools and methods of analysis on the user population and the tasks that they must perform in a given application context. The fi nal purpose is to produce a set of speci ces of more detailed and targeted user requirements.
In fact, many studies have clearly demonstrated that most of the failures of a new technological system can be attributed precisely to the lack of adherence between the characteristics of the product and the improperly identified user requirements. In the case of ProSign, after a careful preliminary analysis, the following functional requirements have been identified:\\
\begin{itemize}
	\item {Profile Management:
		\begin{itemize}
			\item Login (with email, facebook and google);
			\item Logout;
			\item Modify personal information;
			\item Modify passwor;
			\item Forgot password;
			\item Save user email and password;
		\end{itemize}
	}
	\item Learning ASL Management:
	\begin{itemize}
		\item Learn ASL Alphabet;
		\item Learn ASL Words;
		\item Save Words Learned.
	\end{itemize} 
	\item Communication Management:
	\begin{itemize}
		\item Text to Speech;
		\item Speech to Text.
	\end{itemize}
	\item VideoCall Management.
\end{itemize}

\subsection{User acceptance testing}
Once a technological product has reached the final stage of development, it is necessary to verify with appropriate tests if, in the real workplaces where it is destined to operate, positions are detected against the acceptance of the system in terms of services requested to the user and mental predispositions.