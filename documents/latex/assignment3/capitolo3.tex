\rhead{\textit{Study Setting}}
\lhead{}
Before beginning, in this chapter we will describe the steps and the motivations that led us to make certain decisions. To make a usability test, without a proper preliminary organization, it is appropriate to set up a plan to perform these usability tests correctly and in this chapter we will describe the steps and motivations that led us to make certain decisions.\\
Having very general guidelines available, which could inflict part or all of our tests, as general guidelines can turn into overly subjective assessments and dictated by personal tastes, we decided to combine multiple test methodologies so as to obtain results as more reliable and secure. It becomes clear indeed, in front of the discrepancy of the approaches, that the types of rules (among which we mention those of Nielsen, which we have already discussed in the previous chapter) should above all be interpreted and understood, placed in their proper context, adapted to their own objectives. Not all rules adapt to all situations.\\
Important to add that the subjective component can not be excluded a priori, and indeed could be a valuable help to improve the interfaces and usability in general, but there is a need to greatly increase the range of users and testers, so to compare more sensibilities, experiences and possible tastes and to draw up improvements that can make the use experience of this application as much as possible.\\
So we will join the heuristic evaluation, thanks to the help of the Decalogue of Nielsen and the rules dictated by the Material Design, already mentioned in the previous chapter, to get a first redesign and try to solve the first problems of usability.\\
Once this first phase has been completed and once again verified that the improvements made to the application respect the aforementioned heuristics, we will proceed with the tests with the users.

\subsection{Evaluation based on experts}
As already extensively discussed in the previous chapters, we will use the Nielsen Decalogue and the principles of Material Design to draw up an in-depth evaluation, for each of the specs and screens identified in paragraph 2.1.
