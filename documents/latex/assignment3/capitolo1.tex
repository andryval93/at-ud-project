\rhead{\textit{1 INTRODUCTION}}
\lhead{}
With this document we will set up the usability test, analyzing the usability of the ProSign system. The concept of usability is complex to define. We will try to rely on the Nielsen's definition: a measure of the quality of a user's experience interacting with something. A product is usable when it is easy to learn, it allows an efficiency of use, it is easy to remember, it allows few errors of interaction and low gravity, it is pleasant to use.\\
\subsection{Evaluate the usability of an application}
The two most used techniques for evaluating the usability of systems are: the so-called heuristic evaluations and usability tests. In the first case, the evaluation is performed by usability experts, with the help of more or less detailed rules, which reflect the state of knowledge of the sector. The well-known Nielsen heuristics and the guidelines provided by Google, which come under the name of "Material Design", are cited. The latter, while not giving a profound usability evaluation (as these are general guidelines) are useful for our purpose.\\
\subsubsection{Nielsen's Heuristics}
\begin{enumerate}
	\item \textbf{Visibility of system status}: The system should always keep users informed about what is going on, through appropriate feedback within reasonable time;
	\item \textbf{Match between system and the real world}: The system should speak the users' language, with words, phrases and concepts familiar to the user, rather than system-oriented terms. Follow real-world conventions, making information appear in a natural and logical order;
	\item \textbf{User control and freedom}: Users often choose system functions by mistake and will need a clearly marked "emergency exit" to leave the unwanted state without having to go through an extended dialogue. Support undo and redo;
	\item \textbf{Consistency and standards}: Users should not have to wonder whether different words, situations, or actions mean the same thing;
	\item \textbf{Error prevention}: Even better than good error messages is a careful design which prevents a problem from occurring in the first place. Either eliminate error-prone conditions or check for them and present users with a confirmation option before they commit to the action;
	\item \textbf{Recognition rather than recall}: Minimize the user's memory load by making objects, actions, and options visible. The user should not have to remember information from one part of the dialogue to another. Instructions for use of the system should be visible or easily retrievable whenever appropriate;
	\item \textbf{Flexibility and efficiency of use}: Accelerators — unseen by the novice user — may often speed up the interaction for the expert user such that the system can cater to both inexperienced and experienced users. Allow users to tailor frequent actions;
	\item \textbf{Aesthetic and minimalist design}: Dialogues should not contain information which is irrelevant or rarely needed. Every extra unit of information in a dialogue competes with the relevant units of information and diminishes their relative visibility;
	\item \textbf{Help users recognize, diagnose, and recover from errors}: Error messages should be expressed in plain language (no codes), precisely indicate the problem, and constructively suggest a solution;
	\item \textbf{Help and documentation}: Even though it is better if the system can be used without documentation, it may be necessary to provide help and documentation. Any such information should be easy to search, focused on the user's task, list concrete steps to be carried out, and not be too large.
\end{enumerate}

\subsubsection{Material Design}
Material Design is based on three fundamental aspects that we will summarize as follows:
\begin{itemize}
	\item \textbf{Material is the metaphor}:
	Everything is based on the more general idea of Metaphor. Basically you have to try to use abstractions of everyday objects in the creation of your own applications in order to make learning for the end user much easier. Let's take the case of a reader ebook. We tried to make it look like a real book as much as possible, even in the gesture to "flip through pages" so that the user knew a priori how to use this app without the need to learn new concepts. For these reasons also the brightness and the whites take on added value as they serve to highlight elements and make them look as "realistic" as possible.
	\item \textbf{Bold, graphic, intentional}:
	This category includes the so-called "typographical elements". Imagine having to read a written book without the subdivision into chapters or paragraphs, without the use of bold or italics to highlight the keywords, without titles larger than the text and without, in general, the use of white space. Such a book would be unlawful. This example is explanatory for what is meant by this design principle. You must use the intentions, colors, typographical elements in general and white spaces in the best possible way, so as to allow the user easy use of the information and actions to be performed.
	This also serves to better immerse the user in the experience and usability of the app.
	\item \textbf{Motion provides meaning}:
	We must make optimal use of the gesture (which in general are very simple to manage at the code level, thanks to the numerous libraries made available, such as MotionEvent, etc.). The gestures are used to make learning and the use of every day of our apps much easier and faster. Let's take as an example the zoom on a photo. It is much easier to use pinch to zoom than to insert a button that may not be intuitive for our purpose. Another important factor, which is part of this principle, is the use of non-invasive and aggressive feedback. These must accompany the user to always give feedback on the action he is doing and must always be clear and unambiguous. Furthermore, we must always try to use a single environment for carrying out our actions, or if this is impossible, give the impression of this. We must always maintain the integrity of our application, trying to change objects or screens while maintaining continuity and consistency, without sudden changes, abrupt or too excessive.
\end{itemize}
In the second case, the evaluation consists in making users perform precise tasks (Tasks) that simulate the use of the system.