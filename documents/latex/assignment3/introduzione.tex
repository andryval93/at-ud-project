\rhead{\textit{INTRODUCTION}}
\lhead{}
With the term "Deaf Mute" we identify a person who  was either deaf using a sign language or both deaf and could not speak. Nowadays the preferred term is simply "deaf".\\
A deaf person has little to no hearing. Hearing loss may occur in one or both ears. In children, hearing problems can affect the ability to learn spoken language and in adults, it can cause work related difficulties. In some people, particularly older people, hearing loss can result in loneliness. Hearing loss can be temporary or permanent and may be caused by a number of factors, including: genetics, ageing, exposure to noise, some infections, birth complications, trauma to the ear, and certain medications or toxins.\\
Hearing loss is diagnosed when hearing testing finds that a person is unable to hear 25 decibels in at least one ear. Testing for poor hearing is recommended for all newborns. Hearing loss can be categorized as mild, moderate, moderate-severe, severe, or profound. There are three main types of hearing loss: conductive hearing loss, sensorineural hearing loss, and mixed hearing loss.\\
As of 2013 hearing loss affects about 1.1 billion people to some degree. It causes disability in 5\% (360 to 538 million) and moderate to severe disability in 124 million people. Of those with moderate to severe disability 108 million live in low and middle income countries. Of those with hearing loss it began in 65 million during childhood. Those who use sign language and are members of Deaf culture see themselves as having a difference rather than an illness. Most members of Deaf culture oppose attempts to cure deafness and some within this community view cochlear implants with concern as they have the potential to eliminate their culture. The term hearing impairment is often viewed negatively as it emphasises what people cannot do.\\
\subsection*{Signs and symptoms}
\begin{itemize}
	\item difficulty using the telephone;
	\item loss of directionality of sound;
	\item difficulty understanding speech, especially of children and women whose voices are of a higher frequency.
	\item difficulty discriminating speech against background noise (cocktail party effect)
	sounds or speech becoming dull, muffled or attenuated
	\item need for increased volume on television, radio, music and other audio sources. 	
\end{itemize}

Hearing loss is sensory, but may have accompanying symptoms:\\
\begin{itemize}
	\item pain or pressure in the ears;
	\item a blocked feeling
\end{itemize}

There may also be accompanying secondary symptoms:
\begin{itemize}
	\item hyperacusis, heightened sensitivity to certain volumes and frequencies of sound, sometimes resulting from "recruitment"
	\item tinnitus, ringing, buzzing, hissing or other sounds in the ear when no external sound is present
	\item vertigo and disequilibrium
	\item tympanophonia, abnormal hearing of one's own voice and respiratory sounds, usually as a result of a patulous eustachian tube or dehiscent superior semicircular canals
	\item disturbances of facial movement (indicating a possible tumour or stroke)  
\end{itemize}

\subsection*{Sign Languages}
\subsubsection*{LIS}
Italian Sign Language or LIS (Lingua dei Segni Italiana) is the visual language used by deaf people in Italy. Deep analysis of it began in the 1980s, along the lines of William Stokoe's research on American Sign Language in the 1960s. Until the beginning of the 21st century, most studies of Italian Sign Language dealt with its phonology and vocabulary. According to the European Union for the Deaf, the majority of the 60,000–90,000 Deaf people in Italy use LIS.\\
A study of the Department of Information Engineering, Polytechnic University of Marche, Ancona, Italy, has published a paper and related video database related to the LIS. \\
Several other attempts have been made in the literature, but they are typically oriented to international languages (like the American Sign Language - ASL). As in speech, also this kind of language presents different peculiarities strictly depending on the geographical location where it is used. The authors have firstly observed that a specific database for LIS is missing and this shoved them to develop the one here presented. It has been conceived to be used in Automatic Sign Recognition and Synthesis (often referred as Automatic Translation into Sign Languages) applications, which represent an important technological opportunity to augment the social inclusion of people with severe hearing impairments. The Database, namely A3LIS-147, is free and available for download\cite{lisdatabase}. 

\subsubsection*{ASL}
American Sign Language (ASL) is a natural language that serves as the predominant sign language of Deaf communities in the United States and most of Anglophone Canada. Besides North America, dialects of ASL and ASL-based creoles are used in many countries around the world, including much of West Africa and parts of Southeast Asia. ASL is also widely learned as a second language, serving as a lingua franca. ASL is most closely related to French Sign Language (LSF). It has been proposed that ASL is a creole language of LSF, although ASL shows features atypical of creole languages, such as agglutinative morphology.\\
We speak of ASL because it is currently the most widely used sign language and because it is easier to find resources and materials online (such as databases and papers).