\rhead{\textit{Analysis of the Results}}
\lhead{}

The analysis of the results is the conclusive phase of the Usability test. Here we have drawn conclusions on the results obtained and therefore on the problems encountered.

\subsection{Heuristic based analysis}
We have previously used Nielsen Heuristics (as indicated in chapter 1). Below will be named again and for each of these will be highlighted if these are respected or not and why.

\subsubsection{Visibility of System Status}
The ProSign system respects this heuristic thanks to the presence, in each screen, of a clear title and explanatory icons. However, there is a lack of pop-ups or labels that require confirmation of a given action.
\subsubsection{Match between system and the real world}
This heuristic is widely respected thanks to the use of self-explanatory icons that allow a combination of the screen (reachable through that button with that icon) and functionality offered by that screen itself.
\subsubsection{User Control and Freedom}
This heuristic is widely respected thanks, in the first place, to the use of the tabmenu that allows quick and intuitive access to all system functions and thanks to the presence of many combinations or methods to enter information and receive the required results.
\subsubsection{Consistency and standards}
The style and logic of the various screens is consistent with the whole system.
\subsubsection{Error Prevention}
Errors are always prevented thanks to the help of guided choices to obtain certain results. The critical issues have been correctly managed thanks to the help of visual feedback that
show the error and how this needs to be corrected. 
\subsubsection{Recognition rather than recall}
Heuristic totally respected thanks to the aid of schematic and simple to use screens.
\subsubsection{Flexibility and efficiency of use}
This heuristic is not respected as there is only one chance to perform each task.
\subsubsection{Aesthetic and Minimalist Design}
This heuristic is totally respected because every request corresponds to a clear and precise result, without the presence of frills or useless elements.
\subsubsection{Help Users Recognize, Diagnose and Recover from Errors}
This heuristic is fully respected as there are not very understandable error messages or from which it is difficult to find a solution to return to their work.
\subsubsection{Help and Documentation}
Even if the system turns out to be totally usable even without the thorough reading of documentation (since the features are shown in a simple and very usable way), the total lack of documentation causes the non-compliance of this heuristic.

\subsection{Material Design based analysis}
The case of ProSign turns out to be very particular because this application is born as a "Cross Platform" system, that is usable on any operating system. So an analysis based on the Material Design of google, that is oriented mainly to the Android operating system would not be advisable. We have in any case decided to use this parameter as ProSign, despite being a multiplatform application, has many points in common with android applications and lends itself well to this type of analysis. As in the case of Nielsen's Heuristics, also Material Design was presented in Chapter 1 and again the precepts of the Material Design will be renamed and for each of these it will be highlighted whether they are respected or not and why.
\subsubsection{Metaphor}
This precept is widely exploited thanks to the help of icons and images that show and abstract the action that will be performed through that screen.

\subsubsection{Bold, Graphic, Intentional}
This precept is totally observed and information is easily and quickly usable.

\subsubsection{Motion Provides Meaning}
This Precect is fully respected because it makes extensive use of the gestures where necessary (for example to look at the 3D model that explains the ASL alphabet). Moreover, there is always consistency between the various screens because the same graphic elements are always used, in compliance with the design rules.
